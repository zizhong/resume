% !TEX program = xelatex

\documentclass{resume}
%\usepackage{zh_CN-Adobefonts_external} % Simplified Chinese Support using external fonts (./fonts/zh_CN-Adobe/)
%\usepackage{zh_CN-Adobefonts_internal} % Simplified Chinese Support using system fonts

\begin{document}
\pagenumbering{gobble} % suppress displaying page number

\name{Yu Gu}

\basicInfo{
  \email{grain1101@gmail.com} \textperiodcentered\
  \phone{(+1) 669-246-8478} \textperiodcentered\
  \linkedin[Yu Gu]{https://www.linkedin.com/in/yu-gu-2978b0102/}}

\section{\faGraduationCap\ Education}
\datedsubsection{\textbf{Northeast Normal University(NENU)}, ChangChun, Jilin, China}{2012 -- 2015}
\textit{M.S.} in Computer Science(CS)
\datedsubsection{\textbf{Northeast Normal University(NENU)}, ChangChun, Jilin, China}{2008 -- 2012}
\textit{B.S.} in Computer Science(CS)

\section{\faCogs\ Skills}
\begin{itemize}[parsep=0.5ex]
  \item Programming Languages: C++, C#, Javascript, PowerShell
  \item Platform: Windows, Linux
  \item Development: Backend
\end{itemize}

\section{\faUsers\ Experience}
\datedsubsection{\textbf{Microsoft Inc.} Suzhou, China}{2015.07 -- 2016.11}
\role{Software Engineer}{Manager: Mushi Jin}
Backend software engineer working at Microsoft SharePoint team. Main developing lanuages include C#, Powershell.
Platforms include Cosmos, SQL Server, Azure. Project experiences are Project SharePoint Hybrid Taxonomy and Project ODB Cross-Geo Data Move.
\begin{itemize}

\end{itemize}

\datedsubsection{\textbf{Project SharePoint Hybrid Taxonomy}}{2015.07 -- 2016.4}
\role{C#, Cosmos, CSOM, SQL, Restful-like, timer jobs}{}
Hybrid Taxonomy replicates the metadata groups that are mastered in SharePoint Online to on-premises periodically
while keeping the on-premises copies Read Only.
\begin{itemize}
  \item Tree-structured metadata migration.
  \item Designing and developing the main feature, adding unit tests, writing docs.
  \item performance analysis via a Microsoft's internal Bigdata platform Cosmos.
\end{itemize}

\datedsubsection{\textbf{Project ODB Cross-Geo Data Move}}{2016.4 -- 2016.11}
\role{C#, Cmdlets, Odata, Azure} {}
This is a part of SharePoint MultiGeo Project. This is to meet data residency laws,
to save costs and to enrich productivety experience.
\begin{itemize}
  \item Tenant admin triggers requests to another region via Cmdlets and sends to servers through restful API.
  \item Background workflow includes backing up, uploading to Azure, downloading, restoring and cleanup.
\end{itemize}

% Reference Test
%\datedsubsection{\textbf{Paper Title\cite{zaharia2012resilient}}}{May. 2015}
%An xxx optimized for xxx\cite{verma2015large}
%\begin{itemize}
%  \item main contribution
%\end{itemize}


\section{\faInfo\ Miscellaneous}
\begin{itemize}[parsep=0.5ex]
  \item GitHub: https://github.com/grain1101
  \item Linkedin: https://www.linkedin.com/in/yu-gu-2978b0102/
  \item Languages: English - Fluent, Mandarin - Native speaker
\end{itemize}

%% Reference
%\newpage
%\bibliographystyle{IEEETran}
%\bibliography{mycite}
\end{document}
